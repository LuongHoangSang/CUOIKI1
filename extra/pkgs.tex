\usepackage[newparttoc]{titlesec}	% định dạng tiêu đề cho các section
\usepackage{titletoc}	% định dạng mục lục 
\usepackage{pdfpages}	% chèn trang từ file pdf 
\usepackage{lipsum}		% văn bản định dạng trang in 
\usepackage[utf8]{vietnam}	% tiếng Việt 
\usepackage{xparse} % hỗ trợ định nghĩa options cho lệnh tự tạo
\usepackage{xcolor,color,colortbl}  % các gói xử lý màu 
\usepackage{xpatch}	% hỗ trợ định nghĩa lệnh tự tạo 
\usepackage{tkz-tab} % xử lý hình với tikz 
\usepackage{fancybox} % tạo các hộp 
\usepackage[most]{tcolorbox} % định dạng các hộp, khung 
\tcbuselibrary{skins} % thư viện bổ sung cho tcolorbox
\usepackage{graphicx} % Chèn hình, vẽ hình đơn giản 
\usepackage{geometry} % định dạng, canh lề trang in 
\usepackage{indentfirst} % viết hoa đoạn đầu của mỗi mục 
\usepackage{fancyhdr} % tạo header và footer 
\usepackage{longtable} % bảng dài nhiều trang 
\usepackage[locale=DE]{siunitx} % cách viết số đo có đơn vị theo chuẩn DE (gần giống VN) 
\usepackage[T1,T5]{fontenc} % font encoding 
\usepackage{tikz} % gói TikZ vẽ hình 
\usetikzlibrary{decorations.shapes,shapes.geometric,calc,positioning}
\usetikzlibrary{ decorations.markings}

\usepackage[version=3]{mhchem} % công thức và phương trình hóa học 
%\usepackage{chemmacros} % công thức ion trong hóa học
%\usepackage{chemfig} % vẽ cấu trúc hợp chất hữu cơ 
\usepackage{wasysym} % các ký hiệu sinh học, khoa học 
\usepackage{makecell} % hỗ trợ định dạng ô trong bảng 
\usepackage{array} % hỗ trợ định dạng array
\usepackage{amsmath,amssymb} % công thức và ký hiệu toán học 
%\usepackage{mathabx} % ký hiệu toán học bổ sung (+ thiên văn)  - xung đột với wasysym, tránh dùng đồng thời 
\usepackage{enumitem} % định dạng môi trường liệt kê 
\setlist[itemize,enumerate,description]{noitemsep, {nosep}}
\usepackage{cprotect}% cho phép marco hủy tác dụng chống verbatim trong các môi trường tiêu đề 
\usepackage{multicol} % môi trường nhiều cột
\usepackage{environ} % hỗ trợ định nghĩa môi trường
\usepackage{tasks} % hỗ trợ list dạng task 
\usepackage{calc} % hỗ trợ tính toán & đo kích thước văn bản 
\usepackage{multido} % thực hiện lệnh lặp lại 
\usepackage{pgf} % hỗ trợ phép tính toán học và vẽ hình	
\usepackage{setspace} % hỗ trợ định dạng khoảng cách văn bản. 
%\usepackage{showframe}	% hiển thị khung lề khi thiết kế
\usepackage{tabularx} % hỗ trợ bảng 
\usepackage{needspace} % hỗ trợ định dạng khối dòng văn bản 
\usepackage{hyperref}
% định dạng tham chiếu và tham chiếu chéo 
\hypersetup{hidelinks,colorlinks=false,breaklinks=true,bookmarksopen=true}
%\usepackage{slashbox}	% chia chéo trong bảng
%\usepackage{mnsymbol} % tạo thêm symbol (stars)

%\usepackage[varg]{txfonts}	% hỗ trợ font 
%\usepackage{times}			% font Times (kèm txfonts)
%\usepackage{helvet}			% font helvet (không chân - kèm txfonts)

\usepackage{capt-of}
\usepackage{tikz}
\usepackage{multirow}
\usepackage{pgfplots}
\usetikzlibrary{
	pgfplots.fillbetween,
}
\pgfplotsset{compat=1.13}
\usetikzlibrary{decorations.markings,bending}
\usetikzlibrary{shapes.geometric}
\usepackage{tkz-euclide}
\usepackage{fontawesome}
\usepackage{circuitikz}